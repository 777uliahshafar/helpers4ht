\documentclass{article}


\usepackage[english]{babel}
\usepackage{hyperref}
\ifdefined\HCode
\usepackage[T1]{fontenc}
\usepackage[utf8]{inputenc}
\else
\usepackage{fontspec}
\setmainfont{TeX Gyre Schola}
\fi
\usepackage{microtype}

\title{The \texttt{helpers4ht} bundle}
\author{Michal Hoftich\footnote{\url{michal.h21@gmail.com}}}
\date{Version 0.1\\12/11/2015}
\begin{document}
\maketitle
\tableofcontents
This is a bundle of packages providing support for tex4ht configuration. With
the exception of \texttt{alternative4ht}, you shouldn't use them in your
documents, but put them in a \texttt{.cfg} file instead. 

A \texttt{cfg} file is \TeX\ file with special structure, which provides
configurations for  \texttt{tex4ht} conversion. The basic structure is following:

\begin{verbatim}
 ...early definitions...
 \Preamble{options}
 ...definitions...
 \begin{document}
 ...insertions into the header of the html file...
 \EndPreamble 
\end{verbatim}

Options for \verb|\Preamble| commands are the same as for second parameter of 
\texttt{htlatex} command, for example:

\begin{verbatim}
  \Preamble{xhtml,2}
\end{verbatim}


\input{readme}

\input{changelog}
\end{document}
